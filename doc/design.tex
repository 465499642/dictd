% design.tex -- 
% Created: Fri Feb 21 08:04:03 1997 by faith@cs.unc.edu
% Revised: Mon May 26 09:55:53 1997 by faith@acm.org
% Copyright 1997 Rickard E. Faith (faith@cs.unc.edu)
% This program comes with ABSOLUTELY NO WARRANTY.
% 
% $Id: design.tex,v 1.5 1997/05/27 13:07:14 faith Exp $
% 
%
\def\FileCreated{Fri Feb 21 08:04:03 1997}
\def\FileRevised{Mon May 26 09:55:53 1997}

\documentclass{article}
\usepackage{xspace}
\usepackage{ifthen}


\newcommand{\WS}{\mbox{\tt <WS>\ }}
\newcommand{\SP}{\mbox{\tt <SP>\ }}
\newcommand{\NL}{\mbox{\tt <NL>\ }}
\newcommand{\EOF}{\mbox{\tt <EOF>\ }}

% grammar support
\def\leftline{\topsep 0ex\partopsep 0ex\trivlist \raggedright \item[]}
\let\endleftline=\endtrivlist
\newcommand{\note}{\bigskip\par\noindent}
\newcommand{\prd}[3][]{\noindent\begin{leftline}\hspace{1em}{\it #2}\ ::=\ %
  \begin{minipage}[t]{.64\textwidth}\raggedright #3\end{minipage}%
    \ifthenelse{\equal{#1}{}}{}%
    {\begin{minipage}[t]{.54\textwidth}\raggedright #1\end{minipage}%
      \vspace{1ex}}%
    \end{leftline}}
\newcommand{\opt}[1]{[~#1~]}
\newcommand{\optlist}[1]{\leftbrace\ #1\ \rightbrace}
\newcommand{\lhs}[1]{{\it #1\/}}
\newcommand{\leftbrace}{\char'173}  % for cmtt10 font
\newcommand{\rightbrace}{\char'175}
\newcommand{\lit}[1]{{\tt #1}}

\newcommand{\dict}{\textsc{dict}\xspace}
\newcommand{\webster}{\textsc{webster}\xspace}

\begin{document}

\title{The Design and Implementation of the \dict Client/Server}
\author{Rickard E. Faith\thanks{faith@cs.unc.edu}}
\maketitle


\section{Introduction}

This paper was originally intended to provided a complete discussion of the
design and implementation of the \dict protocol.  However, we wrote an RFC
instead that defines the protocol, and re-writing that in this document
doesn't make an sense.  I have, therefore, stopped working on this
document, but will preserve it for historical interest, especially since it
documents the webster protocol.  Later, all the other cruft may be removed
and it may be preserved as a webster protocol reference.

The grammars for the client and server configuration files will be put in
the man pages for those programs, so those sections are probably obsolete
in this document.

\section{Historical Background}

The \webster program has long been available on the Internet.  This program
uses a proprietary Webster's 7th edition dictionary which is available at
only a few sites.  Due to overwhelming load, most of these sites have shut
down access to the general Internet community.\footnote{We believe that
  some original Arpanet sites have a license for a database which allows
  serving the database to other sites, but we have been unable to confirm
  these reports.  We also believe that some NeXT machines contained a copy
  of the same or similar database and that some of the \webster servers on
  the Internet have served this database.} For example, prior to 1994,
generally accessible \webster servers were running at MIT, CMU, Indiana
University, and several other sites.  Since 1996, the only publicly
available server appears to be running at the University of Michigan,
although at least one other server is accessible via the WWW.

Fortunately, in the past few years, several freely-distributable
dictionaries and lexicons have become available on the Internet.  However,
these freely-distributable databases are not accessible via a uniform
interface, and are not accessible from a single site.  Further, they are
often small and incomplete individually, but would collectively provide an
interesting and useful database for providing definitions of English words.
Examples include the Free On-line Dictionary of Computing, the Jargon file,
the WordNet database, and MICRA's version of the 1913 Webster's Revised
Unabridged Dictionary.\footnote{URLs and more information about each of
  these databases are provided in a later section.} Several translating
dictionaries are also available.

\section{The \webster Protocol}

In the interest of preserving historical accuracy, the \webster protocol is
presented here as it was implemented for the University of Michigan server.
The original \webster protocol was able to access a thesaurus database and
probably had additional commands that are not listed in this summary.\note

\prd{command}{\lhs{keyword} \opt{\lhs{space} \lhs{argument}} \NL}
\prd{\SP}{ASCII 32, a space}
\prd{\NL}{ASCII 13 ASCII 10, a CR LF sequence}
\prd{\NL}{ASCII 141, LispMachine NewLine character}
\prd{\EOF}{ASCII 128}
\prd{number}{String of ASCII digits representing a number}
\prd{word}{ASCII string representing a word}

\subsection{The Help Command}

\prd{help}{\lit{HELP} \NL}
\prd{help-response}{\lhs{help-document} \EOF}

\subsection{The Define Command}

\prd{define}{\lit{DEFINE} \SP \lhs{word} \NL}
\prd{define-response}{\lhs{wild-response}}
\prd{define-response}{\lhs{spelling-response}}
\prd{define-response}{\lhs{definition-response}}

\prd{wild-response}{\lit{WILD} \SP \lit{0} \NL}
\prd{wild-response}{\lit{WILD} \NL \lhs{word-list} \EOF}
\prd{word-list}{\lhs{numbered-word}}
\prd{word-list}{\lhs{numbered-word} \lhs{word-list}}
\prd{numbered-word}{\lhs{number} \SP \lhs{word} \NL}

\note A \lhs{wild-response} is given when the word to be defined contained
wildcard characters (`\%' which matches exactly one character, or `*' which
matches 0 or more characters).  The word numbers are unique and can be used
in place of the word itself in a \lhs{define} request.\note

\prd{spelling-response}{\lit{SPELLING} \SP \lit{0} \NL}
\prd{spelling-response}{\lit{SPELLING} \NL \lhs{word-list} \EOF}

\note The \lhs{spelling-response} is returned if the requested word could
not be found in the dictionary.  The list contains possible words which are
within a Levenshtein distance of one from the requested word.  The exact
\webster algorithm is unknown, but it appears that transposition, single
character deletion, single character addition, and single character
correction are all attempted).\note

\prd{definition-response}{\lit{DEFINITION} \SP \lhs{number}
  \NL \opt{\lhs{word-list}} \lhs{definition} \EOF}
\prd{definition}{Any amount of ASCII text}

\note A \lhs{definition-response} is given when an exact match occurs.  The
\lhs{number} given is the number of cross-references for that definition.
These cross-references are given in the \lhs{word-list} (which is empty if
there are zero cross-references given).


\subsection{The Complete Command}

\prd{complete}{\lit{COMPLETE} \SP \lhs{word} \NL}
\prd{complete-response}{\lhs{ambiguous-response}}
\prd{complete-response}{\lhs{completion-response}}

\prd{ambiguous-response}{\lit{AMBIGUOUS} \SP \lhs{number} \EOF}

\note The \lhs{ambiguous-response} is given if the requested word is the
prefix or wild-card match of zero or more than one other words in the
dictionary.\note

\prd{completion-response}{\lit{COMPLETION} \SP \lhs{word} \NL}

\note The \lhs{completion-response} is given if the requested word is the
prefix or wildcard match of exactly one word in the dictionary.  The
matching word is returned in the response.


\subsection{The Endings Command}

\prd{endings}{\lit{ENDINGS} \SP \lhs{word} \NL}
\prd{endings-response}{\lit{MATCHES} \SP \lit{0} \NL}
\prd{endings-response}{\lit{MATCHES} \NL \lhs{word-list} \EOF}

\note The endings command returns a prefix or wild-card match on the
requested word.  The notes specifically state that the numbers returned
cannot be used in a subsequent define command, but the implemented server
appears to allow this.  Note that this command is similar to ``DEFINE
word*''.


\subsection{The Spell Command}

\prd{spell}{\lit{SPELL} \SP \lhs{word} \EOF}
\prd{spell-response}{\lit{SPELLING} \SP \lhs{number} \NL}
\prd{spell-response}{\lit{SPELLING} \NL \lhs{word-list} \EOF}

\note If the requested word matches exactly one word in the dictionary, the
response number is one.  Otherwise it is zero.  Wildcards are not allowed
in spell requests.  If the word is not found, but other words are found
within a Levenshtein distance of one from the requested word, these words
are returned in a list (e.g., centre).

\subsection{Other Responses and Notes}

Additionally, the following error responses are possible:

\prd{error}{\lit{ERROR RECOVERABLE}\lhs{error-message}\NL}
\prd{error}{\lit{ERROR FATAL}\lhs{error-message}\NL}

\note If there is additional data in the response, the data is followed by
an EOF packet (ASCII 128).

After a fatal error, the server will terminate the connection.

\section{Required Database Headwords}

When databases are prepared for a server, they should contain several
headwords that provide information about the database.  Three headwords are
required:
\begin{description}
\item[00-database-info] Human-readable information about the original
  source of the data and about the copyright and licensing information that
  apply to the data.  The information should be sufficient to:
  \begin{enumerate}
  \item Locate the original source of the pristine, non-dict database (and
    to find a potential location of an updated version)
  \item Determine if the data can be freely served or distributed, or if
    restrictions on use or distribution apply to the data
  \item Locate the copyright holder so that any necessary permissions for
    use or distribution can be obtained
  \end{enumerate}
\item[00-database-name] A short (e.g., one line) description of the
  database.  This should be suitable for a credit line printed by a client
  of the form:
  \begin{center}\tt
    From <short-description>:
  \end{center}
\end{description}

The goal here is to make it easy for a human to determine the source,
copyright, distribution, and use restrictions that apply to any given
database.

\section{Available Databases}

In general, dictionary databases that might be useful with \dict fall into
three classes:
\begin{itemize}
\item Databases which are in the public domain or which have licenses which
  specifically permit their use in a program such as \dict.
\item Databases which are accessible via the Internet or the WWW, but which
  have an unknown or ambiguous copyright or other licensing restrictions
  that make their use in \dict problematic.
\item Commercial databases which are available on CDROM, which \emph{might}
  be suitable for personal or educational use but which \emph{cannot} be
  redistributed using the \dict server.
\end{itemize}

Databases in the last two categories are listed here for informational
purposes only.  Many of these databases are probably not suitable for use
with \dict due to legal restrictions on their use and/or distribution.  We
do not have the legal resources available to determine which databases are
redistributable.

The only databases which we will provide for use with \dict are those that
are clearly in the Public Domain, or are otherwise distributable.  This may
mean that we have specific permission from the author or copyright owner to
redistribute the database with the \dict programs.

The example \dict server(s) and client(s) distributed in this package are
all distributed under the GNU General Public License or a less restrictive
license.

However, note that some of the databases that are available in \dict format
may have \emph{additional restrictions} on their use or distribution.  If
you want to redistribute a database, you should determine if this is
allowed by the database owner.  Do \emph{not} assume that the databases are
distributed under terms which are compatible with the \dict distribution
terms.

\section{Server Setup Files}

\subsection{Configuration File}

\prd{server-config-file}{\opt{\lhs{access}} \opt{\lhs{group-list}}
  \lhs{database-list}}
\prd{group-list}{\lhs{group} \opt{\lhs{group-list}}}
\prd{database-list}{\lhs{databse} \opt{\lhs{database-list}}}

\prd{access}{\lit{access} \lit{\{} \lhs{access-list} \lit{\}}}
\prd{access-list}{\lit{allow} \lhs{access-spec} \lhs{access-list}}
\prd{access-list}{\lit{deny} \lhs{access-spec} \lhs{access-list}}
\prd{access-list}{}
\prd{access-spec}{\lhs{user} \lit{@} \opt{\lhs{host-spec}}}
\prd{access-spec}{\lhs{group}\ lit{\#} \opt{\lhs{host-spec}}}
\prd{access-spec}{\lhs{host-spec}}
\prd{host-spec}{\lhs{host-addr-spec}}
\prd{host-spec}{\lhs{host-name-spec}}
\prd{host-addr-spec}{\lhs{number} \lit{.}}
\prd{host-addr-spec}{\lhs{number} \lit{.} \lhs{number} \lit{.}}
\prd{host-addr-spec}{\lhs{number} \lit{.} \lhs{number} \lit{.} \lhs{number}
  \lit{.}}
\prd{host-addr-spec}{\lhs{number} \lit{.} \lhs{number} \lit{.} \lhs{number}
  \lit{.} \lhs{number}}
\prd{host-name-spec}{\lhs{host-wild-spec}}
\prd{host-name-spec}{\lhs{string} \lhs{host-wild-spec}}
\prd{host-wild-spec}{\lit{.} \lhs{string} \opt\lhs{host-wild-spec}}

\prd{group}{\lit{group} \lit{\{} \lhs{user-list} \lit{\}}}
\prd{user-list}{\lhs{username} \opt{user-list}}

\prd{database}{\lit{database} \lhs{db-id} \lit{\{} \lhs{db-spec-list} \lit{\}}}
\prd{db-spec-list}{\lhs{db-spec}}
\prd{db-spec-list}{\lhs{dp-spec-list} \lhs{db-spec}}
\prd{db-spec}{\lit{data} \lhs{filename}}
\prd{db-spec}{\lit{index} \lhs{filename}}
\prd{db-spec}{\lhs{access}}
\prd{db-spec}{\lhs{filter} \lhs{string}}
\prd{db-spec}{\lhs{prefilter} \lhs{string}}
\prd{db-spec}{\lhs{postfilter} \lhs{string}}

\subsection{Password File}

\prd{password-file}{\lhs{entry-list}}
\lhs{entry-list}{\lhs{entry} \opt{\lhs{entry-list}}}

\prd{entry}{\lhs{username}:\lhs{shared-secret}}


\section{Client Setup File}

\prd{client-config-file}{\lhs{server-spec-list} \opt{\lhs{server-group-list}}}
\prd{server-list}{\lhs{server-entry} \opt{\lhs{server-list}}}
\prd{server-group-list}{\lhs{server-group} \opt{\lhs{server-group-list}}}

\prd{server-entry}{\lhs{db-id} \lhs{server-spec} \lhs{shared-secret}}
\prd{server-spec}{\lhs{machine-name} \lit{:} \lhs{port}}
\prd{server-spec}{\lhs{machine-addr} \lit{:} \lhs{port}}

\prd{server-group}{\lhs{group-id} \lit{\{} \lhs{search-list} \lit{\}}}
\prd{search-list}{\lhs{db-id} \opt{\lhs{search-list}}}
\prd{search-list}{\lhs{group-id} \opt{\lhs{search-list}}}
\prd{search-list}{\lit{!} \opt{\lhs{search-list}}}

\end{document}
